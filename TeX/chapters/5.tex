%%%%%%%%%%%%%%
% Robustness %
%%%%%%%%%%%%%%

\section{Robustness}

To test the robustness of the main findings, we estimate a series of fixed-effects models with lagged 
ICT investment variables from 1 to 8 years. The goal is to assess whether the impact of ICT 
investment on unemployment is stable over time and whether delayed effects differ across educational 
groups.

Across all lag structures, the core results hold. ICT investment remains positively and 
significantly associated with unemployment, particularly for low-skilled workers. The size and 
significance of coefficients are remarkably consistent. For example, the effect of a 3-year lagged 
ICT investment on unemployment is 4.846 (***), 2.935 (***), and 1.265 (***) for low-, medium-, and 
high-skilled workers, respectively (see Table~\ref{tab:models_3ylag}).

Interaction effects with welfare regimes are similarly robust. In all lag specifications, ICT 
investment has significantly weaker effects in post-socialist countries (e.g., -5.557*** in the 
3-year lag model for low-skilled) and significantly negative effects for medium- and high-skilled 
workers in Southern European regimes. The Nordic regime shows small but positive effects on 
high-skilled unemployment, potentially reflecting the rapid pace of technological diffusion.

The models include country and year fixed effects, reducing the risk of omitted variable bias. The 
reduction in observations for longer lags is due to the truncation of early years in the panel and 
is not systematically related to the dependent variable, preserving the validity of the estimates.

The consistent direction, magnitude, and statistical significance of the results across all eight lag 
structures confirm that the relationship between ICT investment and unemployment is not driven by 
model specification or short-term fluctuations. These findings support the temporal robustness of the 
digitalization–unemployment link.

Additional robustness checks—such as excluding the global financial crisis years or using alternative 
codings of welfare regimes—are discussed in the Appendix and confirm the stability of the results.
