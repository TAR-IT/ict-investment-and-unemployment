%%%%%%%%%%%%%%%%
% Introduction %
%%%%%%%%%%%%%%%%

\section{Introduction}

Digital transformation and investment in information and communication technologies (ICT) are 
fundamentally reshaping labor markets across advanced economies. While ICT adoption is a key driver 
of productivity growth and innovation \parencite[][p.~49]{oecd2019measuring}, its consequences for 
employment remain contested and unevenly distributed.

Technological change tends to favor highly skilled labor while substituting or transforming routine 
and low-skill tasks \parencite[][p.~1045]{acemoglu2011skills}. This contributes to labor market 
polarization—characterized by rising employment at the high and low ends of the skill spectrum and 
stagnation or decline in medium-skilled occupations \parencite[][p.~1070]{acemoglu2011skills}. ICT 
investments, in particular, reinforce this process by accelerating automation and restructuring job 
content \parencite[][pp.~2--4]{balsmeier2019isthis}, while also enabling new forms of employment 
that combine digital tools with human capital \parencite[][pp.~210--214]{brynjolfsson2014thesecond}.

However, the employment effects of digitalization may not manifest immediately. Firms often require 
time to reorganize production processes or workforce composition following ICT investment. This 
temporal lag in adjustment processes is rarely captured in empirical studies. Moreover, the effects 
of digitalization on unemployment are likely to be moderated by national institutions—especially the 
design of welfare state regimes, which shape labor market resilience and worker reallocation.

This paper investigates how national ICT investment influences unemployment across educational groups 
in OECD countries, taking into account institutional differences and time-lagged effects. It 
addresses the following research question:

\begin{quote}
\textbf{How does national ICT investment influence unemployment rates across educational levels in 
different welfare state regimes over time?}
\end{quote}

The contribution is threefold. First, it examines the heterogeneous effects of digitalization across 
low-, medium-, and high-skilled workers. Second, it models time-lagged relationships to account for 
delayed adjustment processes in labor markets. Third, it explores how welfare state institutions 
condition the impact of ICT investment on unemployment. These insights aim to inform both academic 
debate and policy decisions concerning digital transitions, labor market governance, and skills 
development.
