\section{Introduction}

The increasing digitalization and automation of work are fundamentally transforming labor markets 
worldwide. A key driver of this transformation is investment in information and communication 
technologies (ICT), which contributes significantly to productivity growth and innovation 
\parencite[][p. 49]{oecd2019measuring}.

While technological advances often increase demand for highly skilled workers, the role of 
low-skilled labor remains ambiguous. The distinction between skills and tasks is crucial, as 
technology tends to automate or outsource specific tasks, reshaping job profiles 
\parencite[][p. 1045]{acemoglu2011skills}. This dynamic contributes to labor market polarization: 
employment grows at the high and low ends of the skill distribution, while middle-skill jobs 
decline \parencite[][p. 1070]{acemoglu2011skills}.

ICT investment may accelerate these trends by increasing demand for high-skilled labor while 
displacing lower-skilled tasks \parencite[][pp. 2–4]{balsmeier2019isthis}. Technological change 
not only eliminates jobs through automation but also creates new occupations, especially in 
areas combining human cognitive abilities with digital tools 
\parencite[][pp. 210–214]{brynjolfsson2014thesecond}.

This paper investigates how national ICT investment affects unemployment across different 
educational levels in OECD countries. The study asks:

\begin{quote}
\textbf{“How does national ICT investment influence unemployment rates across educational levels 
in welfare states?”}
\end{quote}

The findings aim to contribute to the debate on digitalization’s labor market effects and provide 
evidence for policy decisions on labor market regulation and education.