%%%%%%%%%%%%%%
% Conclusion %
%%%%%%%%%%%%%%

\section{Conclusion}

This study examined the impact of national ICT investment on unemployment rates across different 
educational groups in 35 OECD and partner countries between 2005 and 2022. Using fixed-effects panel 
models with time lags and interaction terms, the analysis provides three key insights into the 
relationship between digitalization, labor market outcomes, and institutional context.

First, the findings challenge the conventional assumption that digitalization primarily benefits 
high-skilled workers. In contrast to Hypothesis~\textbf{H1}, ICT investment is positively and 
significantly associated with unemployment even among tertiary-educated workers—though the magnitude 
is smaller than for low-skilled groups. This suggests that automation may increasingly affect 
complex cognitive tasks, possibly through advances in artificial intelligence and platform-based work.

Second, consistent with Hypothesis~\textbf{H2}, the strongest and most robust effect of ICT 
investment is observed among low-skilled workers. This confirms the polarization narrative: 
digitalization reinforces structural disadvantages for workers in routine-intensive occupations and 
accelerates labor market segmentation.

Third, institutional context plays a decisive role. Supporting Hypothesis~\textbf{H3}, the effects of 
ICT investment on unemployment are significantly moderated by welfare state regimes. Anglo-Saxon 
countries with flexible labor markets exhibit stronger polarization, whereas post-socialist and 
Southern European regimes show weaker or even negative interaction effects. These results suggest 
that institutional buffers—such as employment protection, training programs, and social insurance—can 
mediate the disruptive impact of digital change.

The robustness checks using time lags from one to eight years confirm the stability of these 
patterns, with the 3-year lag showing the most consistent and interpretable results. Control 
variables behave as expected: GDP per capita is negatively associated with unemployment across all 
education levels, while tertiary education rates and trade union density exhibit more nuanced effects 
depending on skill level.

In sum, this study highlights the need to complement digital transformation with tailored policy 
interventions. Institutional resilience matters: countries that invest in both technology and 
inclusive labor market structures are better equipped to navigate the employment risks of 
digitalization. Future research should further investigate reverse causality and explore micro-level 
mechanisms that connect ICT adoption to job displacement and creation.
