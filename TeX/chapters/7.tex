%%%%%%%%%%%%%%
% Conclusion %
%%%%%%%%%%%%%%

\subsection{Conclusion}

The results of this study provide valuable insights into the relationship between 
\textit{ICT investments} and the \textit{unemployment rate} across different educational 
levels. They reveal significant associations and highlight the role of institutional 
frameworks in shaping the employment effects of digitalization. This research contributes 
to the academic debate on the interactions between technological progress, labor market 
structures, and political institutions, offering practical implications for policymakers, 
businesses, and educational systems.

Hypothesis \textbf{H1}, that \textit{ICT investments} are associated with lower 
\textit{unemployment rates} among highly skilled workers, is not supported by the findings. 
Contrary to expectations, higher \textit{ICT investments} correlate with increased 
unemployment even for the highly qualified, challenging the classical assumption of 
\ac{SBTC} that highly skilled workers generally benefit from digitalization. This may be 
explained by the increasing automation of not only simple but also knowledge-intensive tasks.

Hypothesis \textbf{H2}, that \textit{ICT investments} increase unemployment among 
low-skilled workers, is supported by the results. The strongest positive effect appears 
for the low-skilled group, suggesting that simple jobs are particularly affected by automation. 
This aligns with \ac{SBTC} and job polarization theories, which argue that digitalization 
displaces middle-skill jobs while benefiting high-skill workers.

Hypothesis \textbf{H3}, that institutional factors such as welfare state regimes can mitigate 
the negative effects of \textit{ICT investments}, is partially confirmed. Interaction models 
show that post-socialist and Southern European welfare states dampen the negative employment 
effects of digitalization, whereas Anglo-Saxon countries display a clear positive relationship 
between \textit{ICT investments} and rising unemployment. This indicates that institutional 
structures play a crucial role in how digitalization impacts labor markets.

Control variables provide additional insights. GDP per capita consistently exhibits a 
significant negative effect on unemployment across all education groups. Labor market 
regulation stabilizes employment for the highly skilled but tends to correlate with higher 
unemployment for the low-skilled. The share of tertiary education positively affects 
unemployment in all groups, indicating that increased education alone does not offset 
the negative impacts of digitalization.

Overall, these findings emphasize that the effects of \textit{ICT investments} are highly 
context-dependent and intertwined with institutional frameworks. While digitalization leads 
to job losses in some countries, it can stabilize or even foster employment in others. This 
calls for comprehensive policy approaches combining technological investments with labor 
market and education policies to manage structural change socially and effectively.
