%%%%%%%%%%%%%%%%%%%%
% Data and Methods %
%%%%%%%%%%%%%%%%%%%%

\section{Data and Methods}

\subsection{Data Sources}

This study uses data from the OECD, which provides harmonized economic and social statistics 
across countries \parencite{oecd2025}. The main variables are ICT investments 
\parencite{oecd2022ict} and unemployment rates by educational attainment 
\parencite{oecd2022unemployment}. Control variables include GDP per capita 
\parencite{oecd2022gdp}, trade union density \parencite{oecd2022tud}, tertiary education share 
\parencite{oecd2022education}, and employment protection regulation 
\parencite{oecd2022regulation}. Welfare state type is classified following Esping-Andersen 
\parencite{espingandersen1990thethree}.

The dataset covers 35 OECD and selected non-OECD countries\footnote{Australia, Austria, Belgium, 
Bulgaria, Brazil, Canada, Croatia, Czechia, Denmark, Estonia, Finland, France, Germany, Greece, 
Hungary, Iceland, Italy, Ireland, Latvia, Lithuania, Luxembourg, Netherlands, New Zealand, 
Norway, Poland, Portugal, Romania, Spain, Sweden, Switzerland, Türkiye, Slovak Republic, 
Slovenia, United Kingdom, United States.} from 2005 to 2022, resulting in 3973 observations after 
merging and cleaning.

ICT investments (share of GDP) capture gross fixed capital formation in digital infrastructure, 
software, and technologies. Unemployment rates are disaggregated by educational attainment: low 
(no or lower secondary education), medium (upper secondary or vocational training), and high 
(tertiary education).

Missing values for trade union density, tertiary education share, and labor market regulation 
were linearly interpolated or extrapolated. GDP per capita was rescaled (per 1000 USD) for 
interpretability.

\subsection{Operationalization}

The dependent variable is the unemployment rate (\texttt{UNEMPLOYMENT\_RATE\_PERCENT}), by 
education level. The main independent variable is ICT investment 
(\texttt{ICT\_INVEST\_SHARE\_GDP}), defined as the share of GDP invested in ICT assets 
\parencite{oecd2022ict}.

Control variables:
\begin{itemize}
  \item \textbf{GDP per capita} (\texttt{GDP\_PER\_CAPITA}): measures economic prosperity 
  \parencite{oecd2022gdp}.
  \item \textbf{Trade union density} (\texttt{PERCENT\_EMPLOYEES\_TUD}): captures collective 
  bargaining strength \parencite{oecd2022tud}.
  \item \textbf{Tertiary education share} (\texttt{PERCENT\_TERTIARY\_EDUCATION}): proxy for 
  human capital \parencite{oecd2022education}.
  \item \textbf{Labor market regulation} (\texttt{REGULATION\_STRICTNESS}): degree of employment 
  protection \parencite{oecd2022regulation}.
  \item \textbf{Welfare state type} (\texttt{WELFARE\_STATE}): Nordic, Central European, 
  Anglo-Saxon, Southern European, Post-socialist \parencite{espingandersen1990thethree}.
\end{itemize}

\subsection{Analytical Strategy}

We apply fixed-effects (FE) panel models to estimate the impact of ICT investment on unemployment 
by education level. FE models control for unobserved country-specific heterogeneity and focus on 
within-country variation over time. Random-effects models are not used due to potential 
correlation between unobserved heterogeneity and explanatory variables 
\parencite{wooldridge2010econometric}.

We include year fixed effects to account for macroeconomic shocks and technological trends. 
Interaction terms between ICT investment and welfare state type allow us to assess institutional 
moderation of ICT effects on labor markets.
