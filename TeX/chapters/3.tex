%%%%%%%%%%%%%%%%%%%%
% Data and Methods %
%%%%%%%%%%%%%%%%%%%%

\section{Data and Methods}

\subsection{Data Sources}

This study employs a balanced panel dataset from 2005 to 2022, combining harmonized macroeconomic and 
labor market data from the OECD. The core variables include:

\begin{itemize}
  \item \textbf{ICT investment} as a share of GDP, based on gross fixed capital formation in 
  information and communication technology assets \parencite{oecd2022ict},
  \item \textbf{Unemployment rates} disaggregated by educational attainment (low, medium, high) 
  \parencite{oecd2022unemployment},
  \item \textbf{Control variables} comprising GDP per capita \parencite{oecd2022gdp}, trade union 
  density \parencite{oecd2022tud}, tertiary education share \parencite{oecd2022education}, and labor 
  market regulation strictness \parencite{oecd2022regulation}.
\end{itemize}

Countries are categorized into five welfare state regimes: Nordic, Central European, Anglo-Saxon 
(liberal), Southern European, and Post-socialist, based on Esping-Andersen’s classification and 
ubsequent literature \parencite{espingandersen1990thethree, ferrera1996thesouthern}.

The sample includes 35 OECD and selected non-OECD countries\footnote{Australia, Austria, Belgium, 
Brazil, Bulgaria, Canada, Croatia, Czechia, Denmark, Estonia, Finland, France, Germany, Greece, 
Hungary, Iceland, Ireland, Italy, Latvia, Lithuania, Luxembourg, Netherlands, New Zealand, Norway, 
Poland, Portugal, Romania, Slovak Republic, Slovenia, Spain, Sweden, Switzerland, Türkiye, United 
Kingdom, United States.}. After merging and cleaning, the final dataset contains 3,973 observations. 
Including lagged versions of ICT investment leads to a reduction in usable observations due to 
missing initial years in the time series; this is a standard consequence in dynamic panel models and 
is addressed in robustness checks.

Missing values for the control variables were imputed using linear interpolation (for internal gaps) 
and last-value extrapolation (for initial/final years). GDP per capita was rescaled (in 1,000 USD 
units) for interpretability.

\subsection{Operationalization of Variables}

\textbf{Dependent variable:}  
Unemployment rate by education level (\texttt{UNEMPLOYMENT\_RATE\_PERCENT}) serves as the outcome 
variable. Education levels are categorized according to ISCED standards:  
\begin{itemize}
  \item \textit{Low} = less than upper secondary education (ISCED 0–2),
  \item \textit{Medium} = upper secondary or post-secondary non-tertiary (ISCED 3–4),
  \item \textit{High} = tertiary education (ISCED 5–8).
\end{itemize}

\textbf{Main independent variable:}  
ICT investment (\texttt{ICT\_INVEST\_SHARE\_GDP}) is expressed as a percentage of GDP. To capture 
temporal dynamics, the analysis includes time-lagged versions of this variable from 1 to 8 years 
(e.g., \texttt{ICT\_INVEST\_SHARE\_GDP\_L3}).

\textbf{Moderating variable:}  
Welfare state regime (\texttt{WELFARE\_STATE}) is a categorical variable interacting with ICT 
investment to test for institutional moderation effects.

\textbf{Controls:}
\begin{itemize}
  \item GDP per capita (\texttt{GDP\_PER\_CAPITA}): economic development level.
  \item Tertiary education share (\texttt{PERCENT\_TERTIARY\_EDUCATION}): human capital availability.
  \item Labor market regulation (\texttt{REGULATION\_STRICTNESS}): protection level for employment.
  \item Trade union density (\texttt{PERCENT\_EMPLOYEES\_TUD}): collective bargaining capacity.
\end{itemize}

All variables are mean-centered and standardized where appropriate for comparability.

\subsection{Analytical Strategy}

To estimate the causal effect of ICT investment on unemployment, we apply fixed-effects (FE) panel 
regression models separately for each education group. FE models control for all time-invariant 
unobserved heterogeneity across countries (e.g., geography, cultural norms), isolating within-country 
variation over time \parencite{wooldridge2010econometric}.

We include:
\begin{itemize}
  \item Country fixed effects,
  \item Year fixed effects to capture global shocks (e.g., financial crisis, COVID-19),
  \item Interaction terms between ICT investment and welfare state type.
\end{itemize}

Our core specification focuses on a 3-year lag between ICT investment and labor market outcomes, 
based on theoretical considerations and model fit. Additional lag structures (1–8 years) are 
estimated as robustness checks and discussed later.

Standard errors are clustered at the country level to account for serial correlation. All models are 
estimated using the \texttt{plm} package in R with robust variance estimators.