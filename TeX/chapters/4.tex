%%%%%%%%%%%
% Results %

%%%%%%%%%%%

\section{Results}

\subsection{Descriptive Results}

Table~\ref{tab:summary} presents descriptive statistics for the key variables across the sample of 
35 OECD and selected partner countries (2005–2022). ICT investments range from 0.7\% to 8.7\% of GDP, 
with a mean of 2.5\%. Unemployment rates vary substantially, with a mean of 8\% and values ranging 
between 0.8\% and 50\%. GDP per capita shows significant dispersion, ranging from 13.3k to 137.7k 
USD (mean: 43.7k USD). Trade union density and tertiary education shares also display considerable 
cross-country heterogeneity.

\begin{table}[H]
\centering
\caption{Descriptive statistics (2005–2022)}
\label{tab:summary}
\begin{tabular}{lcccc}
\toprule
Variable & Mean & SD & Min & Max \\
\midrule
ICT investment (\% of GDP) & 2.46 & 0.98 & 0.73 & 8.69 \\
Unemployment rate (\%) & 7.95 & 6.34 & 0.82 & 49.89 \\
GDP per capita (1000 USD) & 43.73 & 17.13 & 13.34 & 137.72 \\
Union density (\%) & 28.45 & 20.71 & 4.50 & 92.20 \\
Tertiary education share (\%) & 33.65 & 9.27 & 12.87 & 59.96 \\
Labor regulation (0-6) & 2.19 & 0.83 & 0.00 & 4.88 \\
\bottomrule
\end{tabular}
\end{table}

Country-level trends reveal substantial differences. For example, Sweden shows persistently high ICT 
investment (4–5\% of GDP) and low unemployment among high-skilled workers. Spain illustrates stronger 
cyclical effects, with low-skilled unemployment peaking during crises, while ICT investment increased 
moderately over time. Poland experienced declining unemployment across all skill groups alongside 
relatively stable ICT investment. These patterns suggest that macroeconomic context and institutional 
settings, rather than ICT investment alone, drive short-term employment dynamics.

Figure~\ref{fig:plots} illustrates these relationships for selected countries representing different 
welfare regimes. The plots highlight variation in unemployment trajectories by education and modest 
correlations with ICT investment levels.

\begin{figure}[H]
\centering
\caption{ICT investment and unemployment trends by education in selected countries}
\label{fig:plots}
\end{figure}

\subsection{Descriptive Results}

\section{Multivariate Results}

Table~\ref{tab:models_control} summarizes the fixed-effects models with controls. ICT investment 
shows a positive and significant association with unemployment across all education groups: 
low-skilled (2.302***), medium-skilled (1.157***), and high-skilled (0.455***). The strongest 
association is found for low-skilled workers, supporting the hypothesis that digitalization 
disproportionately affects routine and low-skill jobs.

\begin{table}[H]
\centering
\caption{Fixed-effects models with controls}
\label{tab:models_control}
\begin{tabular}{lccc}
\toprule
 & Low skill & Medium skill & High skill \\
\midrule
ICT investment & 2.302*** & 1.157*** & 0.455*** \\
GDP per capita & -0.194*** & -0.153*** & -0.083*** \\
Tertiary share & 0.606*** & 0.282*** & 0.140*** \\
Union density & 0.128*** & 0.106*** & 0.025+ \\
Labor regulation & -0.147 & -0.118* & -0.085* \\
R$^2$ & 0.304 & 0.308 & 0.272 \\
\bottomrule
\end{tabular}
\end{table}

ICT investments are associated with higher unemployment in flexible labor markets (Anglo-Saxon 
countries). Interaction models (Table~\ref{tab:models_interaction}) show that institutional settings 
significantly moderate these effects. In post-socialist states, ICT investment effects on 
unemployment are significantly weaker compared to Anglo-Saxon regimes (e.g., -5.200*** for 
low-skilled). In Southern European states, ICT investments are linked to lower unemployment for 
medium- and high-skilled workers. Nordic states show a small positive interaction for high-skilled 
unemployment (+0.639*), possibly reflecting fast digital transitions that challenge even highly 
educated workers.

\begin{table}[H]
\centering
\caption{Interaction models: ICT investment $\times$ welfare regime}
\label{tab:models_interaction}
\begin{tabular}{lccc}
\toprule
 & Low skill & Medium skill & High skill \\
\midrule
ICT investment (base: Anglo-Saxon) & 4.671*** & 3.246*** & 1.259*** \\
$\times$ Post-socialist & -5.200*** & -3.579*** & -1.415*** \\
$\times$ Central European & -0.871 & -0.594 & -0.302 \\
$\times$ Nordic & 0.220 & -0.317 & 0.639* \\
$\times$ Southern European & -1.801 & -3.066*** & -2.880*** \\
R$^2$ & 0.337 & 0.333 & 0.306 \\
\bottomrule
\end{tabular}
\end{table}

Macroeconomic controls, such as GDP per capita, remain robust and negative across models, indicating 
that stronger economies are associated with lower unemployment. The inclusion of welfare regime 
interactions increases the explanatory power (R$^2$) by about 3-5 percentage points, highlighting 
the relevance of institutional context in shaping the employment effects of digitalization.

Overall, the findings suggest that ICT investments contribute to labor market polarization but that 
institutional factors can buffer or amplify these effects depending on the welfare regime.
