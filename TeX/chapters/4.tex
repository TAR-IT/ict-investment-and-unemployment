%%%%%%%%%%%
% Results %
%%%%%%%%%%%

\section{Results}

\subsection{Descriptive Summary and Variable Overview}

\input{assets/variables_summary_final.tex}

Table~\ref{tab:variables_summary} provides an overview of key variables used in the analysis. ICT 
investment ranges from 0.7\% to 8.7\% of GDP across the sample, with a mean of 2.5\%. Unemployment 
rates vary substantially across educational levels and countries, with a sample-wide average of 
approximately 8\%. GDP per capita displays wide variation, from 13,300 to over 137,000 USD (mean: 
43,700 USD), reflecting the heterogeneity of national economic contexts. Education, labor market 
regulation, and union density also show significant cross-country variation.

\begin{table}[H]
\centering
\caption{Overview of the distribution of welfare state types}
\centering
\begin{tabular}[t]{lrr}
\toprule
Kategorie & Anzahl & Prozent\\
\midrule
Anglo-Saxon & 2382 & 19.98\\
Central European & 2943 & 24.69\\
Nordic & 2034 & 17.07\\
Other & 0 & 0.00\\
Post-socialist & 3015 & 25.30\\
\addlinespace
Southern European & 1545 & 12.96\\
\bottomrule
\end{tabular}
\label{tab:variables_welfarestate}
\end{table}


Welfare states are categorized based on Esping-Andersen’s typology and later refinements to account 
for post-socialist and Southern European models. The classification used in this study is summarized 
in Table~\ref{tab:variables_welfarestate}, which underpins the interaction terms in the multivariate 
models.

\subsection{Baseline Models without Lag}

\input{assets/models_nolag_final.tex}

Table~\ref{tab:models_nolag} displays the fixed-effects panel regressions without temporal lags. ICT 
investment is positively and significantly associated with unemployment across all education groups: 
4.671 (***), 3.246 (***), and 1.259 (***) for low-, medium-, and high-skilled workers, respectively. 
These results suggest an immediate adjustment process in the labor market, particularly for 
occupations exposed to automation and digital substitution.

The interaction effects reveal that institutional structures significantly mediate the relationship. 
Post-socialist regimes exhibit strongly negative interactions—most notably for low-skilled workers 
(-5.200***), indicating a protective institutional environment. Southern European regimes mitigate 
unemployment risks for medium- and high-skilled workers. Nordic countries show a minor but positive 
interaction for high-skilled workers (+0.639*), potentially reflecting transition frictions.

Control variables behave as expected. GDP per capita is consistently associated with lower 
unemployment, while tertiary education share and trade union density correlate positively, likely 
capturing structural differences across economies.

\subsection{Core Estimates: 3-Year Lag Models}

\input{assets/models_3ylag_final.tex}

Table~\ref{tab:models_3ylag} presents the fixed-effects models with a 3-year time lag between 
ICT investment and unemployment. Across all educational groups, ICT investment remains positively 
and significantly associated with unemployment: 4.846 (***), 2.935 (***), and 1.265 (***) for low-, 
medium-, and high-skilled workers, respectively. These results confirm the hypothesis that 
digitalization disproportionately affects workers engaged in routine tasks, with the strongest 
effects observed among the low-skilled.

Interaction terms between ICT investment and welfare regime types reveal substantial heterogeneity. 
In post-socialist countries, the relationship between ICT investment and unemployment is 
significantly weaker than in liberal regimes, with interaction effects of -5.557 (***), -3.494 (***), 
and -1.543 (***) across the three skill levels. Southern European countries exhibit significant 
negative interactions for medium- and high-skilled unemployment, indicating a mitigating effect. 
Nordic regimes show a small but positive interaction for high-skilled workers (+0.593, +), possibly 
reflecting short-term adjustment frictions in rapidly digitalizing labor markets.

Control variables perform as expected. GDP per capita is robustly negative and significant, 
indicating that economic prosperity is associated with lower unemployment. The share of tertiary 
education and trade union density are positively associated with unemployment, likely reflecting 
structural labor market features in highly educated economies. Employment protection strictness has 
weakly negative or insignificant effects.

The inclusion of welfare regime interactions enhances the explanatory power (R$^2$) of the models by 
3–5 percentage points compared to baseline specifications without institutional context. This 
suggests that national institutional frameworks play a critical role in shaping how digital 
transformation impacts labor market outcomes.

\subsection{Model Comparison and Temporal Interpretation}

The comparison between the no-lag and 3-year lag models reveals several important insights. While the 
sign and significance of core coefficients remain stable across specifications, the magnitude of the 
ICT effects is slightly larger in the lagged models. For instance, the effect for low-skilled 
unemployment rises from 4.671 to 4.846. This suggests that the labor market consequences of ICT 
investment unfold with a temporal delay, rather than instantaneously.

The interaction terms with welfare regimes are directionally consistent but tend to be more 
pronounced in the lagged models. This implies that institutional features exert a growing influence 
over time, as digitalization gradually reshapes employment patterns and labor market regulations come 
into play.

Overall, the shift from contemporaneous to lagged specification improves model fit (as indicated by 
R$^2$ and AIC/BIC), and strengthens the interpretation that structural unemployment effects of 
digitalization materialize over multiple years.

\subsection{Temporal Dynamics and Interaction Patterns}

Although the 3-year lag serves as the primary analytical model, robustness tests with alternative lag 
structures (1 to 8 years) confirm the direction and stability of results. Main and interaction 
effects remain significant and consistent, although their magnitude peaks between years 2 and 4. This 
supports the notion that digital labor market disruptions emerge gradually, with institutional 
mediation playing an increasingly important role over time.

The next section presents a more detailed robustness analysis, including the full set of alternative 
lag models.
