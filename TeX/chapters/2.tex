%%%%%%%%%%%%%%%%%%%%%%%%%%%%%%%%%%%%%
% Related Literature and Hypotheses %
%%%%%%%%%%%%%%%%%%%%%%%%%%%%%%%%%%%%%

\section{Related Literature and Hypotheses}

\subsection{Digitalization, ICT Investment, and Labor Markets}

Digitalization and automation have deeply transformed labor markets, particularly affecting 
employment structures across skill levels. Routine-intensive tasks—both manual and cognitive—are 
increasingly automated, with low- and medium-skilled jobs most at risk of displacement 
\parencite{frey2013thefuture, goos2014explaining}. Simultaneously, demand has risen for 
high-skilled workers with analytical and technological competencies 
\parencite{autor2013thegrowth}. This phenomenon is often described as job polarization 
\parencite{autor2015whyare}.

ICT investment is a key driver of this transformation. Studies show that ICT-intensive firms gain 
efficiency and competitiveness, but these benefits are unequally distributed across the workforce 
\parencite{corrado2018intangible, brynjolfsson2014thesecond}. While creating new jobs in digital 
sectors, ICT investment often accelerates the automation of routine tasks, putting low-skilled 
workers at greater risk of unemployment.

\subsection{Welfare State Institutions and Labor Market Polarization}

Institutional frameworks shape how countries adapt to technological change. Welfare state regimes 
differ in their capacity to mitigate adverse effects of digitalization. Nordic welfare states, 
with strong labor market policies and active training systems, may better cushion job 
polarization \parencite{espingandersen1990thethree}. Liberal regimes, emphasizing market 
flexibility and minimal social protection, may experience stronger polarization 
\parencite{hall2001varieties}. Central European and Southern European regimes present mixed 
patterns, depending on their regulatory structures and inclusiveness 
\parencite{ferrera1996thesouthern}.

\subsection{Theoretical Framework}

This study builds on Schumpeter’s concept of \textit{creative destruction}, where technological 
progress disrupts existing structures but enables long-term economic renewal 
\parencite{schumpeter1976capitalism}. Theories of \textit{skill-biased technological change} 
(SBTC) and \textit{routine-biased technological change} (RBTC) further explain how digitalization 
increases demand for high-skilled labor while eroding routine jobs 
\parencite{violante2008skill, goos2014explaining}.

\subsection{Hypotheses}

Based on the literature, we propose three testable hypotheses:

\begin{itemize}
    \item \textbf{H1:} Countries with higher ICT investment exhibit lower unemployment rates 
    among high-skilled workers.
    \item \textbf{H2:} Higher ICT investment is associated with higher unemployment among 
    low-skilled workers, due to the automation of routine tasks.
    \item \textbf{H3:} Welfare state regimes moderate the polarization effect. Nordic regimes 
    exhibit less labor market polarization, while liberal regimes show stronger polarization.
\end{itemize}
