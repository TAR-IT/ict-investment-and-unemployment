%%%%%%%%%%%%%%%%%%%%%%%%%%%%%%%%%%%%%
% Related Literature and Hypotheses %
%%%%%%%%%%%%%%%%%%%%%%%%%%%%%%%%%%%%%
\section{Related Literature and Hypotheses}

\subsection{ICT Investment, Digitalization, and Labor Market Polarization}

Digitalization and ICT adoption are reshaping labor markets through task automation and structural 
shifts in employment. A growing body of literature shows that routine-intensive jobs—whether manual 
or cognitive—are particularly susceptible to automation \parencite{frey2013thefuture, 
goos2014explaining}. This technological substitution disproportionately affects low- and 
medium-skilled workers, while demand for high-skilled labor increases due to complementarities with 
new technologies \parencite{autor2013thegrowth, autor2015whyare}. These dynamics underpin the widely 
observed trend of labor market polarization across OECD countries.

ICT investment is a critical enabler of this transformation. Firms investing in ICT typically 
experience higher productivity and innovation output, but also undergo organizational restructuring 
that alters labor demand \parencite{corrado2018intangible, brynjolfsson2014thesecond}. While ICT may 
generate employment in high-skill digital sectors, it often displaces workers in routine occupations, 
particularly where upskilling opportunities are lacking. Consequently, the net employment effect of 
ICT depends on both technological characteristics and institutional context.

\subsection{Welfare State Institutions and the Moderation of ICT Effects}

The labor market consequences of digitalization are not uniform across countries. National 
institutions, especially welfare state regimes, influence how technological disruptions affect 
employment. Following Esping-Andersen’s typology \parencite{espingandersen1990thethree}, Nordic 
regimes provide extensive social protection and active labor market policies, which may buffer 
adverse shocks and facilitate workforce adaptation. In contrast, liberal regimes (e.g., Anglo-Saxon 
countries) emphasize market flexibility and minimal intervention, potentially exacerbating job 
losses among vulnerable groups \parencite{hall2001varieties}.

Southern and Central European regimes occupy intermediate positions, characterized by fragmented or 
conservative welfare systems \parencite{ferrera1996thesouthern}. Post-socialist regimes present 
another variant, often marked by weaker institutional capacities but also distinct labor market 
legacies from planned economies. These institutional variations likely condition the extent to which 
ICT investment translates into labor market polarization.

\subsection{Theoretical Framework}

This study draws on three complementary strands of theory. First, Schumpeter’s concept of 
\textit{creative destruction} highlights the dual nature of technological change: it displaces 
existing structures while enabling long-term renewal and productivity gains 
\parencite{schumpeter1976capitalism}. Second, the theory of 
\textit{skill-biased technological change} (SBTC) explains rising wage inequality and employment 
growth among high-skilled workers, as new technologies complement their capabilities 
\parencite{violante2008skill}. Third, \textit{routine-biased technological change} (RBTC) refines 
this argument by emphasizing that routine tasks—regardless of skill level—are especially prone to 
automation, leading to job polarization \parencite{goos2014explaining}.

Taken together, these frameworks suggest that the employment effects of ICT are both heterogeneous 
and mediated by institutional settings. The temporal dimension also matters: firms often implement 
technological changes gradually, suggesting lagged effects of ICT investment on labor market outcomes.

\subsection{Hypotheses}

Based on the literature and theoretical framework, we formulate the following hypotheses:

\begin{itemize}
    \item \textbf{H1:} Higher national ICT investment is associated with 
    \textit{lower unemployment rates among high-skilled workers}, reflecting complementarity with 
    digital technologies.
    
    \item \textbf{H2:} Higher ICT investment is associated with 
    \textit{higher unemployment among low-skilled workers}, driven by automation of routine tasks.
    
    \item \textbf{H3:} \textit{Welfare state regimes moderate} the effect of ICT investment on 
    unemployment. Nordic regimes mitigate polarization effects, while liberal regimes amplify them.
\end{itemize}
