%%%%%%%%%%%%%%%%%%%%%
% Policy Discussion %
%%%%%%%%%%%%%%%%%%%%%

\section{Policy Discussion}

The findings confirm that national ICT investment significantly influences labor market outcomes 
across educational groups, with stronger adverse effects on low-skilled workers. However, these 
effects are not uniform. The interaction terms demonstrate that institutional 
configurations—particularly welfare state regimes—play a critical role in shaping the employment 
consequences of digitalization.

Countries with liberal labor markets, often characterized by low employment protection and minimal 
retraining infrastructure, experience stronger polarization effects. In contrast, post-socialist and 
Southern European regimes appear to absorb or mitigate some of the labor market shocks induced by ICT 
investments. This indicates that social institutions can buffer digital disruption—especially when 
automation pressures are high.

For policymakers, three implications follow:

\begin{enumerate}
    \item \textbf{Strengthening labor market institutions:} Active labor market policies (ALMPs), 
    such as job search assistance, retraining programs, and employment subsidies, may be critical to 
    reduce digitalization-induced unemployment, particularly among low- and medium-skilled workers.
    
    \item \textbf{Targeted education and reskilling:} Expanding access to tertiary education and 
    offering modular reskilling paths for workers in declining occupations can help match labor 
    supply with shifting demand.
    
    \item \textbf{ICT-specific inclusion strategies:} Since digital investments are expected to rise, 
    inclusive digital transformation strategies should address not only infrastructure but also labor 
    market integration—especially in countries where automation risks are high and institutional 
    capacity is limited.
\end{enumerate}

Overall, the analysis underscores that technological change is not exogenous to policy. Welfare 
institutions can either amplify or cushion the inequality effects of digitalization. Cross-national 
variation in ICT-induced unemployment suggests that digital policy must be accompanied by robust 
social policy to ensure equitable outcomes.
